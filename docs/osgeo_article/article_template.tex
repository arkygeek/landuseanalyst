% Template GRASS newsletter - Article Language: Latex LA This
% project aims to define a settlement's use of, and impact on, their
% surrounding landscape.

% Head

\title{LA} \subtitle{Identifying Landuse Catchments for 
ancient settlements.}

\author{by Jason Jorgenson and Tim Sutton}

\maketitle

\section{Introduction} \label{sec:Introduction}

This paper describes a software application we are developing that automates
the process of computing landuse requirements for a study being conducted in
the Jordan Valley. The Jordan Valley is found in the Southern Levant, which is
a name commonly used to refer to the geographic region broadly described as
modern-day Israel, Palestine and Jordan, between the Dead Sea and Lake
Tiberias (see Figure \ref{fig:shunaGoogleEarth}).

\begin{figure}[htbp] %Location
\includegraphics[scale=0.17]{./images/ShunaGoogleEarth3D.jpg}
% img42.jpg: 800x600 pixel, 72dpi, 28.22x21.17 cm, bb=0 0 800 600
\label{fig:shunaGoogleEarth} \caption{Looking South-South-East from Lake
Tiberias Down the Jordan Valley towards Shuna.} \end{figure}

We describe how the application logic for Landuse Analyst (LA) was prototyped as a
BASH shell script calling various GRASS utilities, and then developed into a
more flexible application with a graphical user interface.

\section{Background} \label{sec:Background} For an archaeologist, understanding
the relationship that existed between people in the past and their landscape is
very important.  People throughout time have relied on various plants and
animals, wild or domestic, to provide them with the food they require to
survive.  Regardless of the period in time being examined, a relationship
exists between people and the space which they occupy.  A better understanding
of this relationship can help to more confidently theorise about a range of
issues in the reconstruction of past societies.  Imagine, for example, the
relationship that nomadic hunter gatherers had with the land, and then compare
that to  an artisan living in a settlement of several thousand people.  We can
piece together clues revealed through a combination of archaeological surveys,
ethnoarchaeology, written records, oral traditions, and excavations. The
insight gained from these clues can offer a more concrete understanding of the
relationship people had with their landscapes, allowing archaeologists to
better understand topics such as social and economic organisation\footnote{In
one sense, economic organization can be thought of as the strategy used to
collect and distribute resources.  In modern societies, urban centres rely upon
the surplus production of agricultural commodities provided by farmers in the
countryside.  This reliance necessitates a certain level of organization
between rural and urban people.  By applying this same logic in historical
contexts, we can infer landuse patterns .}, as well as short and long term
impacts communities might have had on the landscape.  An understanding of long
term impacts of ancient humans on their landscape can be useful for modern day
researchers by, for example, providing insight into the potential environmental impact of
current human activities.

\section{Evidence} \label{sec:Evidence} By examining seeds, bones, and other
items found at archaeological sites, archaeologists are able to determine 
dietary habits. From the relative quantities found estimates can be made of the
importance each food source would have had in the diet.  Specific agricultural practices
can be detected in the archaeological record in a number of ways.  If the scale
of production is known, crop selection, for example, can help indicate whether
crops were being cultivated for subsistence use or for more commercial
purposes\footnote{Sometimes, written accounts exist which can help to
piece this together.}.  If there is evidence of surplus production of any
particular crop or animal, this can be a possible indication that it was being
grown commercially.  Other possible inferences include taxation, storage,
redistribution, indications of intensified farming, and craft specialization,
all of which can indicate increased economic complexity.  Through careful
study, a better understanding of the land use habits and patterns of these
populations can unlock many interesting facets of the culture and lifestyle of
these peoples.

\section{Early attempts} \label{sec:EarlyAttempts} 

In order to learn more about how people interacted with their land, it is very
helpful to determine an approximate area of land to support settlement
populations and it's likely distribution.  In order to do this, it is necessary
to first know how much land would have been needed to produce enough food to
sustain the settlement.  A three step process is used to calculate the amount of
land a settlement would have required based on it's population and diet
make-up.  

XXXXX What are the three steps???  XXXXXXXXX
YYYYY THEY ARE:  YYYYY :P

\section{Calorie targets}
  \index{LA!Calorie Targets}
  Calorie Targets are the first level of calculations done by LA.  Calorie targets are obtained through a number of steps (see Fig. \ref{fig:dietDiagram}), which are as follows.

\subsection{Steps in calculating calorie targets}

\begin{figure}[ht] %dietDiagram
 \centering
 \includegraphics[scale=.20]{./images/dietDiagram.jpg}
 % dietDiagram.jpg: 740x731 pixel, 72dpi, 26.10x25.78 cm, bb=0 0 740 731
 \textit{\caption{\label{fig:dietDiagram}How LA looks at a settlement's diet composition.}}
\end{figure}

\begin{figure}[ht] %dietMeatDiagram
 \includegraphics[scale=.20]{./images/fancyDietMeat.jpg}
 % dietDiagram.jpg: 740x731 pixel, 72dpi, 26.10x25.78 cm, bb=0 0 740 731
 \textit{\caption[Plant Portion of Diet]{\label{fig:dietMeatDiagram}Specification of meat component of diet in LA.}}

\end{figure}

\begin{figure}[ht] %dietCropDiagram
 \centering
 \includegraphics[scale=.2]{./images/dietFancyCrops.jpg}
 \textit{\caption[Plant Portion of Diet]{\label{fig:dietCropDiagram}Specification of the plant component of diet in LA.  Note that the input figures for percent of diet were: \textbf{HorseBean} $7\%$, \textbf{Einkorn} $13\%$, \textbf{Emmer} $34\%$, \textbf{Lentils} $12\%$, \textbf{Peas} $15\%$, \textbf{Olives}\ $18\%$.  These numbers translate to those shown above after processing as in Fig. \ref{fig:dietDiagram}.}}
\end{figure}

    \begin{enumerate}
      \item \textbf{\textit{Basic Information}} - The user supplies LA with the population of the settlement, as well as the average daily calorific requirements of an average member of the population.  By multiplying these figures together, the total number of calories required for the settlement is calculated (see Fig. \ref{fig:LADiet}).

      \item \textbf{\textit{Primary Dietary Components}} - This step works on the principle that calories can come from only two fundamental sources: plants and animals.  By adjusting a slider right or left, this overall ratio is set.  To explain how this works, let's say that the yearly calorific requirement for a settlement is $1,000,000$.  If the slider is set to show $70\%$ Plants and $30\%$ meat, this would mean that LA will calculate the number of calories that need to come from Plants as $700,000$ and those that need to be supplied by meat as $300,000$.

      \item \textbf{\textit{Detailed Dietary Components}} - At this step, the process gets split into two sections.  One section is for the Plant portion of the diet, and the other is for the Meat portion.  Again, the calculations are done based on values which are set by a slider that indicates what percentage of the plants or animals come from domestic sources.  To illustrate this, we will continue using the figures above of $700,000$ for plants and $300,000$ for animals.  If the slider on the plant side is set to $90\%$ Tame\footnote{The word tame was used purely for aesthetic purposes in the design of the input form of LA} then the number of calories that need to be supplied by domestic plants (crops) would be $700 \cdot 90\%$ or $630,000$ calories.  As an example, if peas are defined as a crop, and the percentage of diet setting in Wheat's Parameters is set to $15.00\%$ then using the values from the previous step, LA calculates that Peas must provide $630,000 \cdot 15.00\%$ or $94,500$ calories.  This also means that Peas represent $15\%\cdot90\%\cdot70\%=9.45\%$ of the overall diet.  Remember that the $15\%$ is a portion of the \textit{tame crop} part of the diet! ( See Fig. \ref{fig:dietCropDiagram} and Fig. \ref{fig:dietMeatDiagram}).  The exact same thing happens with meat.

      \item \textbf{\textit{Crop/Animal Contributions}} - The final step is easy to understand, but is not done in the Diet section of the program.  Rather, it is calculated using values that are supplied as parameters for each individual crop and animal.  This parameter is set with a spin-box in the Parameter Manager form.  This value tells LA that this individual animal or crop comprises the indicated percentage of the calories being supplied by tame sources, which is the value from the previous step.
    \end{enumerate}

\begin{figure}[htbp]
  \label{fig:LADiet}%LADiet
    \includegraphics[scale=.4]{./images/LanduseAnalystDiet545.jpg}
    % LanduseAnalystDiet545.jpg: 638x417 pixel, 762dpi, 2.13x1.39 cm, bb=0 0 60 39
  \caption{ Diet Tab, LA}
\end{figure}

\section{Production targets}
  \index{LA!Production Targets}
  Production targets are measures of weight, and represent how many Kg of each crop and animal must be produced to satisy the Calorie Targets calculated above.  Because crops and animals need to be treated somewhat differently at this stage, I will deal with each in a seperate section, beginning with Crops as they are most easily explained.

  \subsection{Crops}
  Another peice of data which the user must supply is food value of each crop.  Using the calorie targets and the user supplied food value, the model calculates the production target necessary to meet these needs.  For example, if you need $337,500$ calories of wheat, and wheat has a food value of $3000$ calories/Kg then you need $\frac{337500}{3000}$ or $112.5$ Kg of wheat.

  \subsection{Animals}
  Although the first part of the calculations for Production Targets of Animals is identical to that of Crops, there is one further consideration.  Animals get slaughtered, and only part of their carcasses are usable as food.  This usable part of their live weight is supplied by the user when they define the animal in the Animal Manager form as what percentage of their live weight is usable as food.  To illustrate this simply, let's say that $50\%$ of a goat's live body weight is usable as food.  If the first part of the production target calculations resulted in a target for goat meat of $100$ Kg, then this last stage of the production target calculation process would change this to $200$ Kg by dividing $100$ by $50\%$   $\left[ \frac{100}{50\%}=200\right]$.

\section{Area targets}
\index{LA!Area Targets}
Now that we have production targets, it is possible to calculate area targets for each crop and animal by looking at yield values for plants, and grazing requirements for animals.  As with Production Targets, Area Targets are quite simple to compute for crops, but get very complicated for animals.

Animals are difficult to calculate area targets for because their land requirements are largely based on their numbers.  On top of this, it isn't just the number of animals needed for slaughter that is required; you also have to remember that a herd of adult females large enough to sustain a steady supply of offspring with numbers enough to keep the production targets met has to be added into the equation as well.  In addition to this, animals can graze fallow crop land, eat fodder from the crops being grown, as well as graze other types of land.  The details of how this all works is explained in detail later.  For now, lets look at how area targets are calculated for crops.

  \subsection{Crops}
    A list of all crops that have been defined is found when the Crops tab is clicked (Fig. \ref{fig:crop}).  Note that the user has the
choice of selecting it for inclusion in the model, as well as being able to select different parameters from a drop-down list.  To the
right of the drop down list is the crop's contribution to the tame plant portion of the diet using the currently selected parameter.  The total of these percentages must be equal to $100\%$ before the model will run, and the current total is always visible in the bottom left
hand corner of the form.

    \begin{figure}[htbp]
        \includegraphics[scale=.4]{./images/LanduseAnalystCrops546.jpg}
      \label{fig:crop} \caption{Crops Tab, LA Version 0.1 Revision 544}
    \end{figure}
    
  \textbf{Crop rotation}\index{LA!Crop Rotation} affects the area target directly, so a few points about it need to be
mentioned.  Quite simply, LA  treats crop rotation as a ratio of sown land to fallow land.  If a ratio of $1:1.00$ is used,
this means that the practise was to sow their land every other year and leave last years cropland fallow (More complicated crop rotations
are possible in LA, and the process of setting this up are explained in more detail in the Help section of the program itself).  The crop
rotation data is entered as a crop parameter in the Crop Parameter form (See Fig. \ref{fig:cropParameters} on page
\pageref{fig:cropParameters})  by adjusting the value of a spinbox either with the mouse wheel, the $+/-$ buttons, or manually entering a
value with the keyboard.
    \begin{figure}[htbp] %crop parameters
        \includegraphics[scale=.27]{./images/cropParameters.jpg}
        % cropParameters.jpg: 842x384 pixel, 762dpi, 2.81x1.28 cm, bb=0 0 80 36
      \caption[Crop Parameters]{\label{fig:cropParameters}\textit{Crop Parameters form for setting the particulars of the crop.}}
    \end{figure}


  \subsection{Animals}
  Herd Size\index{LA!Grazing Requirements} determination is perhaps the single most difficult calculation that LA
performs.  Added to the complexity of this is the problem that different animals have different dietary requirements, and even further, even if they are the same, they require different amounts of feed depending on things like whether they are pregnant or lactating, and what the weather conditions are like.  Again, to further complicate this issue, not all land that can be grazed has the same food value to the animals.

  With the exception of weather conditions, LA addresses all of the issues.  One of the first things a user must do when using
Landuse Analust is to define all of the crops and animals that the settlement used.  By define, I mean that they must provide information about many aspects of them, including (for animals) information relating to their dietary requirements, their reproduction cycle, and their grazing preferences.  Most importantly however, \textit{TDN} must be explained.

    \subsubsection{TDN}
      \label{TDN}
      \index{TDN}
      \index{Total Digestable Nutrients}
    \textbf{TDN} is an acronym for \textbf{T}otal \textbf{D}igestable \textbf{N}utrients, and is a commonly used term among ranchers to describe the quality of the feed available to their animals, and is expressed as a measure of weight.  To help illustrate this, I will use an example.  A rancher buys bales of hay that weigh exactly the same from two different sources.  He discovers that he needs 15 bales a day to feed his herd when using the bales from the first supplier, and only 12 a day when using those from the other.  The TDN or Total Digestable Nutrients of the hay from the second supplier is thus obviously higher than that from the first.  The quality of the hay is better.  This same concept can be applied to land, where a hectare of one type of land might provide 1000 kg of TDN the same area of land with different cover or other characteristics might only be able to provide $850$ kg of TDN.  The nice thing about working with TDN is that it is a universal kind of measure.  TDN can be used for any grazing animal, and since the purpose of LA is to identify the land which a settlement was using for food production, this is all we need.  For a complete description of how to define animals, please see  \textit{Defining Animals}, section \ref{definingAnimals} on page \pageref{definingAnimals}.

--------------------------


Once an area target is known, the land surrounding the
site must be classified as either suitable or unsuitable for each crop sown and
each type of animal being raised.  Lastly, the area target from the first step
can be used to query the land suitability classification map for that area of
land surrounding the site.

From a GIS point of view, one of the greatest challenges of this project was
developing a method for finding the outer extent (or boundary), within which
the combined area of suitable land satisfies the target area.  The usable land
surrounding a settlement is almost certainly not going to contiguous, but
rather be comprised of multiple, irregularly shaped polygons. A further
complication is the fact that some of the irregularly shaped polygons may
be bisected one or more times by the 'boundary line'.  

In late 2006, a BASH script was created running in a GRASS shell.

This script found a specific area of classified, suitable land surrounding a
point (Fig.\\ref{fig:euclideanResults})  by starting at a point and moving
outwards in a circle five meters at a time, checking each time to see if there
was enough suitable land contained with the perimeter.  As soon as it was equal
to or greater than the target area, the loop was ended, and the solution was
deemed found.  An except from the script (shown below) illustrates how 
the target area was calculated:

\begin{smallverbatim} 
while [ ${area} -le ${land_reqrd_total} ]; 
do 
  ((radius=${radius} + 5)) 
  r.circle -b output=circle coordinate=\ 
  744800,3611100 max=\${radius} --overwrite 
  g.remove rast=catchment 
  r.mapcalc catchment=``${src}''*circle 
  rm /tmp/catchment_report 
  r.stats -an inpu=catchment > /tmp/catchment_report 
  cat /tmp/catchment_report 
  area_dec=`awk '{print $2}' /tmp/catchment_report` 
  echo "a catchment area with a radius of \ $radius}
        meters contains area=${area_dec//.*/}"
  echo "${area} square meters of land and target is" 
  echo ${land_reqrd_total} square meters"
done 
\end{smallverbatim}

  


\section{Moving to a graphical user interface} \label{GUI} 
 
The complexity of the animal and crop modelling made using BASH scripts very
awkward.  While the BASH scripting aproach worked well for a proof of concept,
the implementation included a large number of hard coded variables and the solution 
did not provide a flexible environment for experimentation. For example, adding 
and removing different types of crops and animals to the analysis required modifictions 
to the BASH script each time. The BASH script approach approach also had poor reporting
capabilities and offered little 'hand holding' to the user as the analysis was 
carried out.

Consequently we embarked on the development of a Graphical User Interface (GUI) based 
application (see Fig. \ref{fig:htbp}), written in C++ and Qt4. This programming environment decision was largely 
motivated by a desire to capitalise on existing code bases from openModeller
\footnote{openModeller is available at \url{http://openmodeller.sourceforge.net/}} 
and Quantum GIS (QGIS)\footnote{QGIS is available at \url{http://qgis.org}}.  
For invoking GRASS tools we opted to use QProcess to launch GRASS commands in 
their own process and then implemented application logic to parse the results of 
each command from stdout.

\begin{figure}[htbp] %Proof of Concept
  \includegraphics[scale=0.37]{./images/LanduseAnalyst544.jpg}
   % img42.jpg: 800x600 pixel, 72dpi, 28.22x21.17 cm, bb=0 0 800 600
   \label{fig:euclideanResults} \caption{LA's primary interface}
   \end{figure}


\section{Analytical functionality} \label{sec:Analytical Functionality}

LA includes various routines for calculating the amount of land 
the settlement needs.  Briefly, these can outlined as follows:

  \begin{enumerate} 
  \item Breaking down the basic diet of the community into specific crops and
  animals being used for nourishment.  Each individual crop and animal needs to
  be expressed as a percentage of the peoples' diet.  This is estimated using
  proportions based on archaeological evidence such as faunal and
  paleobotanical remains discovered from excavations.  
  \item Calculating  calorie targets for the crops and animals identified above.  
  \item Calculating production targets which fulfill the calorie targets.
  Production targets are calculated in kilograms, and take into account factors such as 
  calories per kilogram of produce, and what percentage of an animals weight is
  usable as food.  
  \item Calculating land area targets needed to satisfy the
  production targets.  Each crop and animal is allocated a specific area
  target.  
  \item Carrying out a spatial analysis of the land surrounding the settlement to
  find land suitable for each crop and animal that satisfies the area targets.  
  \end{enumerate}

The details of how these targets are determined are a major part of Landuse
Analyst, but beyond the scope of this article.  What will be discussed in
further detail is how the land classification and identification problems were
approached.

\section{Catchment creation methods} 

When searching for land that meets the area targets, LA starts at
the coordinates of the settlement and moves outward to a point where the land
contained within equals the area target.  LA implements three
different methods for searching outwards from the settlement.  The key
difference between the three methods is the way in which a cost-surface gets
generated, which we will now examine.  All three methods currently used by LA
require a \textbf{D}igital \textbf{E}levation \textbf{M}odel (DEM) to generate
the cost surfaces.



  \subsubsection{Euclidean} \label{subsection:Euclidean} 
  
  This method ignores all topographic features of the landscape when moving
  outwards from the site.  Essentially we are drawing circles around the site,
  with the site right in the middle.  Drawbacks to using this method exist.
  Whilst moving across a landscape, people are affected by slope, rivers,
  landcover, etc.  Circles are just convenient when there is no easy way to
  calculate with an alternative method.  The primary reason for its inclusion
  as a catchment area creation method in LA is to provide the user
  with the option of seeing the difference between this rather simplistic
  approach and the other two methods.

  \subsubsection{Path Distance}\index{Path Distance}
  
  Path Distance is a cost surface creation method that looks at elevation data
  from a DEM and calculates distance from the site taking into account the
  extra distances travelled going up or down slopes.

  \subsubsection{Walking Time}\index{Walking Time}
  
  Walking Time is likely going to be the primary choice of the three methods of
  creating catchment areas in LA.  The cost surface that is
  generated for this method (see Fig. \ref{fig:rwalk}) uses a DEM to calculate
  how long it would take to walk from a starting point (the site, or ZERO on
  the cost surface) to all points on the DEM within a five hour walk ($18,000$
  seconds). Five hours is simply the time used for this case
  study\footnote{This number is currently hard coded. Future versions will
  allow the user to edit this value.} This creates a cost surface with values
  from $0$ to $18,000$ (See Fig. \ref{fig:rwalk}).  The GRASS module r.walk is
  used to produce this cost surface.

%\setkeys{Gin}{width=1\textwidth}
\begin{figure}[htbp] \centering
\includegraphics[scale=0.29]{./images/rwalkShuna.jpg}
   %caption of the figure
   \caption{Cost surface generated by GRASS module
   \textit{r.walk}} \label{fig:rwalk} \end{figure}

\section{Land suitability raster masks} 

The land suitability masks must be binary rasters, meaning that the cells of
the raster file only contain 0 or 1 (NULL values are currently not supported).
The land that is deemed suitable for use should be set to a value of 1, and the
rest of the land is set to 0.  This binary raster  can then be multiplied by a
selection layer (as created by either Walking Distance, Path Distance,
or Euclidean methods).  This selection layer (also a binary raster) grows in size until enough area is
found within it's bounds - identifying the land to be analysed.

In the current version, the software gives the option using three different
methods for landuse classification.  They are: 

\begin{enumerate} 

\item  Use only the user supplied classification map (a binary mask) 

\item  Use minimum and maximum slope values to create a classification map
which will be used exclusively.  This can be done for common crop land and
common grazing land independantly.  For example, the user might stipulate
$0^\circ \leq m \leq 9^\circ$ for crops and $9^\circ < m \leq 15^\circ$ for
grazing land, where $m=slope$.  

\item  Using a combination of the two classification maps.  Slope can be chosen
to either add to or subtract from the user supplied map, or alternatively, the
user supplied map can either add to or subtract from the slope map.  This
combination of the two can be very useful if, for example, a user wishes to use
a soil classification map as the primary indication of landuse suitability, but
refine that map by taking out the land they consider too steep for use.  In
this case the program generated slope maks would be subtracted from the user
supplied (soil) classification map.  

\end{enumerate}


XXXXXX Why use BASH script output at this point XXXXXXXXXX


  \begin{figure}[htbp] %Location
  \includegraphics[scale=0.225]{./images/landcatchment.jpg}
   % img42.jpg: 800x600 pixel, 72dpi, 28.22x21.17 cm, bb=0 0 800 600
   \label{fig:landCatchment} \caption{Results of the BASH script which
   identified the suitable land surrounding a point equal to a particular area.
   The different shades represent different settlement populations.}
   \end{figure}

\section{Finding the land} 

In order to find the suitable land required
to produce enough food to sustain the settlement that is in closest proximity, 
a conditional loop is used which defines the outer extent of the catchement area, 
and then calculates the area of suitable land contained within it.  This process is identical for all
analysis methods.  The earliest versions of LA started at the
closest point to the site that could potentially solve the problem
\footnote{The closest point, or minimum radius, is a perfect circle equal in
area to the target.}, and then moved steadily outwards until the area target
was found.  The amount to move outward in each step was a value provided by the
user, and might have been a value like 30 (which in the case of walking
distance meant 30 seconds, on  a cost surface of 20,000.  This would mean that
potentailly, there would have to be $\frac{20,000}{30}$ or nearly 7000 iterations!)
The basic steps involved were:

   \begin{enumerate} 
   
   \item  Set the initital boundary to start the analysis
   
   \item  Calculate the area of suitable land found within this boundary 
   
   \item If the area of land required has been satisfied, write the results to a new
   file for that item and exit the loop 
   
   \item  Increase the value of the analysis boundary 
   
   \item  Repeat the loop (starting at Step 2)
   
   \end{enumerate}

This method proved extremely time consuming;  If even 3,500 loops, which is
about half of the cost surface map, was required, and each loop took as little
as 45 seconds, the computation time was nearly 44 hours.   To increase the
efficiency, a modified binary search is now used.  This involved a somewhat
different approach.  Instead of using a step amount, a percentage is used.
This percentage value is called the \textit{Precision}. 
The area of land deemed as equalling the current area target gets changed into
a range $\pm$ the precision value multiplied by the target area. 
If a value of $5\%$ is entered, and the area
target is 100 hectares, LA accepts $100 \pm(\frac{100 \cdot
5\%}{2})$ (which is $97.5$ to $102.5$ hectares) as the target area.  The loop
uses three terms, $CurrentMidValue$ $FirstValue$ and $LastValue$ At the
beginning of the process, $FirstValue=0$ and $LastValue=18,000$ which coincides
with the full extent of the cost surface.  $CurrentMidValue$ is calculated
within the loop.  The loop works like this:

  \begin{enumerate} 
  
  \item Set the analysis boundary: $CurrentMidValue=\frac{FirstValue +
  LastValue}{2}$.  
  
  \item Calculate the area of suitable land found within the boundary whose
  outermost boundary is $CurrentMidValue$ 
  
  \item If the contained area falls within the target range, write the results
  to a new file and exit the loop 
  
  \item If the contained area is more than the maximum value in the target
  range, set a halfway between $FirstValue$ and $CurrentMidValue$. This is done
  by making $LastValue=CurrentMidValue$ and then returning back to step number
  one.  (Note $FirstValue$ remains unchanged.) 
  
  \item If the contained area is less than the minimum value in the target
  range, set a halfway between $LastValue$ and $CurrentMidValue$. This is done
  by making $FirstValue=CurrentMidValue$ and then returning back to step number
  one.  (Note $LastValue$ remains unchanged.) 
  
  \end{enumerate}

By using this method, the \textit{maximum} number of steps in the loop is 17.
This method of searching for the area targets made a huge difference in Landuse
Analyst's speed\footnote{ a typical 'run' on a decent computer system is
$\approx$ 2 to 3 minutes now, compared to up to two weeks using the original
method.}, allowing users to now examine many different variations of an entire
model quickly and easily even on an average computer system.  The resultant
maps clearly show which land was likely to have been used, and for what purpose
it was being used for\ref{fig:caseStudy}
 
  \begin{figure}[htbp] %Location
  \includegraphics[scale=0.225]{./images/LEB130007030FallowSlope.jpg}
  \label{fig:caseStudy} \caption{Case Study: Shuna  Population 3000, EB1a}
  \end{figure}

\section{Future development} \label{FuturePlans} 

LA is still in the early stages of its development and many
improvements to usebility and aplication capabilities are planned. That said,
the application is already capable of producing useful results (though more
testing is still needed) in a more flexible, user friendly and efficient manner
than the original BASH prototype.

There are many routes that LA can take from this point with
respect to it's future development; it can continue on as a standalone
application, or turn into a plugin for another application like openModeller or
QGIS.  Before these decisions are made, however, the features in the current
version must be finalised and implemented.  Some features which are currently
in the planning phase are network analysis  for inter-site relationships,
experiment settings, where the model can be set to run a given number of times
using different input variables, and report generation complete with graphs and
maps. 

The network analysis feature will look at all contemporaneous sites in a given
area, and examine production potential.  If a site can potentially produce
excess meat, but falls short in cropland, it will look to neighboring sites to
see if they can make up the difference.  If there is a potential for trade or
supplementation, the most efficient walking routes can then be found using
r.drain on a cost surface generated with r.walk.  This could potentially show
the likely pathways between sites, as well as how intensively used they might
have been.  Taking this further, one might look for places where these routes
merge, or cross, which might be an indicator for a potential archaeological
site which has yet to be discovered.

The current version of LA requires that every scenario be modelled
manually and seperately.  For example, one might wish to look at how adjusting
the yield of crops would affect the land requirements to simulate drought
years.  One might also want to compare the results of different dietary
proportions of meat content to plant content.  With the experiment module, it
will be possible to have the software automatically cycle through all of these
different scenarios automatically.

Drawing much on the work done in openModeller, there is also the hope that it
will be possible to have LA compile presentation quality reports,
complete with spreadsheets and graphs.  This will be a great time saving
feature.  In addition, this will provide a consistency that will make it much
easier to systematically compare results with other users.

Anyone interested in knowing more about this project, or better yet, in
contributing to it, please don't hesitate in contacting us.


%etc.  \end{smallverbatim}


%if you want to cite please use:

% \cite{name:year} or \citep{name:year}

%.... revealed by \cite{herborg:2003} if it shall be in parentheses use
%\citep{herborg:2003}


%End of text.



\begin{footnotesize}
%\begin{thebibliography}{99}

%\bibitem[Herborg et~al. (2003) Herborg, Bentley, Clare, Rushton]{herborg:2003}
%L.M. Herborg, M.G. Bentley, A.S. Clare, S.P. Rushton (2003) \newblock The
%spread of the Chinese mitten crab (Eriocheir sinensis) in Europe; the
%predictive value of an historical data set.  \newblock {\em Hydrobiologia}
%503: 21-28.


%\end{thebibliography}
\end{footnotesize}


\address{Jason Jorgenson\\ University of Liverpool\\
\url{http://www.arkygeek.com}\\ \email{arkygeek AT gmail com}}

\address{Tim Sutton\\ Centro de Referncia em Informao Ambiental, CRIA\\ \url{http://cria.org.br}\\
\email{timlinux@linfiniti.com}}

%%% Local Variables:
%%% mode: latex
%%% TeX-master: main_document.tex
%%% End:

